\subsection*{Learning Objectives}
	Students need to be able to\ldots
	\begin{itemize}
		\item Model optimization problems. \\[-20pt]
		\item Recognize what type of optimization problem they need.\\[-20pt]
		\item Use \texttt{Jupyter Notebook} to approximate the solution.\\[-20pt]
		\item Create a visualization of the problem and the solution.
	\end{itemize}

\subsection*{Context}
	
In lecture we have been studying different types of optimization problems:
\begin{itemize}
	\item Unconstrained optimization
	\item Constrained optimization (Lagrange multipliers)
	\item Linear programming
	\item Calculus of Variations (Euler-Lagrange equations)
\end{itemize}

We have also introduced some extra ways to study optimization problems:
\begin{itemize}
	\item Sensitivity of the objective function with respect to a parameter
	\item ``Shadow-cost'' and the meaning of Lagrange multiplier values
\end{itemize}

In this class, we want to study new problems and make use of these extra ideas to study how the problem and its solution changes when we change parameters. \\


In this tutorial, we will do the following:
\begin{itemize}
	\item Define what we want to optimize
	\item Model and approximate the solution using \texttt{python} tools
	\item Visualize the solutions
	\item Assess the solutions
\end{itemize}


\paragraph{Important.} Don't rush problem 1 to be able to get to problem 2. Even if the students only have time to think about defining the problem and setting assumptions for problem 2, that's ok!


\subsection*{Resources for TAs}

Some \texttt{Jupyter Notebook} scripts:

\begin{itemize}
	\item Boardgame problem \ref{q1}: \href{https://utoronto.syzygy.ca/jupyter/user-redirect/git-pull?repo=https://github.com/bigfatbernie/IBLMathModeling&subPath=tutorials/tutorial2/boardgame-linearprog.ipynb}{\tt boardgame-linearprog.ipynb}
	\item Elevator problem \ref{q2}: \href{https://utoronto.syzygy.ca/jupyter/user-redirect/git-pull?repo=https://github.com/bigfatbernie/IBLMathModeling&subPath=tutorials/tutorial2/elevators.ipynb}{\tt elevators.ipynb}
\end{itemize}


\subsection*{Before Tutorial}

Send an announcement to students letting them know that they will need to bring a laptop and will be using Jupyter Notebooks \url{https://utoronto.syzygy.ca}.


\subsection*{What to Do}
	
Introduce the learning objectives for the day's tutorial. \\

Have students get into small groups and start on \#1 -- each group needs to have at least 1 laptop. 

The first problem should be very self explanatory. Students should be able to do it without hints. \\

On part (b), students should be able to report on how many games go into each shelf instead of just showing with the solution vector is.\\

On part (c), observe that depending on where we put the defective shelf, the computer will give us different results -- some clearly not optimal. 

Students should be aware that even though some problems are ``symmetric'', the methods we use to approximate might not be, so they should plan to check solutions that in theory should be the same.


\hfill 

The second question is open ended and students will probably struggle to define a problem to optimize.

Some key ways to help students:
\begin{itemize}
	\item Students should think of a problem related to the elevator function that they \textbf{CAN} optimize -- this means they'll have to make simplifying assumptions.
	\item Hint \#1: Assume a ``static'' elevator problem, where the elevators are in a position (to be optimized) waiting to be called
	\item Hint \#2: What kinds of trips do people in an apartment building make? -- look towards simplifying assumptions
\end{itemize}

Be open to students having different models and different objectives, as long as they are actionable.

	


%\subsection*{Notes}
%
%	\begin{enumerate}
%		\item Note
%	\end{enumerate}
	
	

	
	
