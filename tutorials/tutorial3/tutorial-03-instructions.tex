\subsection*{Learning Objectives}
	Students need to be able to\ldots
	\begin{itemize}
		\item Model with ODEs. \\[-20pt]
		\item Use Jupyter Notebook to approximate the solution. \\[-20pt]
		\item Create a visualization of the problem and the solution.
	\end{itemize}

\subsection*{Context}
	
In lecture we started studying some modelling with ODEs. 
In this tutorial, we will practice some more modelling and acting on the models to understand the phenomena better.


\paragraph{Important.} Don't rush problem 1 to be able to get to problem 2. 


\subsection*{Resources for TAs}

Some \texttt{Jupyter Notebook} scripts:
	\href{https://utoronto.syzygy.ca/jupyter/user-redirect/git-pull?repo=https://github.com/bigfatbernie/IBLMathModeling&subPath=tutorials/tutorial3/tutorial3.ipynb}{\tt tutorial3.ipynb}


\subsection*{Before Tutorial}

Send an announcement to students letting them know that they will need to bring a laptop and will be using Jupyter Notebooks \url{https://utoronto.syzygy.ca}.


\subsection*{What to Do}
	
Introduce the learning objectives for the day's tutorial.

Have students get into small groups and start on \#1 -- each group needs to have at least 1 laptop. \\

Problem \ref{q1}(c)-(d) are hard. Using the Desmos \url{https://www.desmos.com/calculator/xxswtammtc}, and letting $t$ increase, we can observe the dashed line that connects the amount of pollen the bee has connected to the moment the bee left the previous flower.
The average rate of pollen gathering is the slope of that line, so we can see that the line gets steeper and steeper until the moment when it becomes tangent to the orange ``pollen'' curve. At that point, it is more advantageous for the bee to stop collecting pollen and to move to the next flower.

\textit{Hint for students. } How can you visualize the average rate of pollen gathering in the drawing? 


	
\hfill 



In Exercise \label{q2}, here is some analysis of units for the ideal gas equation: $\rho = \dfrac{Mp}{RT}$:

\begin{itemize}
	\item $101.325$ J $  = 1$ atm L \quad and \quad $1$ atm $= 101325$ Pa
	\item $\left[\dfrac{Mp}{RT}\right] = \dfrac{kg}{J \cdot Pa} = \dfrac{kg}{1000 L} = \dfrac{kg}{m^3} = [\rho]$
\end{itemize}



%
%	
%	rho = Mp/RT  with  p in Pa
%	
%	
%	p = atm = 101.325 kPa
%	
%	101.325 J = 1 atm L
%		
%	Mp / RT = kg / J  * Pa
%			= 101.325 kg / ( atm L ) * atm / 101325
%			= kg / (1000L)
%	rho	= kg / m^3
%			
%	




%\subsection*{Notes}
%
%	\begin{enumerate}
%		\item Note
%	\end{enumerate}
	
	

	
	
