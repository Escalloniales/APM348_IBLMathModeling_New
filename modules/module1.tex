\Heading{Modeling}

Suppose you are observing some \emph{green} ants walking on the sidewalk.
In the first minute you record 10 ants. In the second minute you
record 20. In the third minute, you record 40 ants. This continues until
there are too many ants for you to count.

\begin{center}
	\begin{tabular}{c|c}
		Minute & \#Green Ants\\
		\hline
		1 & 10\\
		2 & 20\\
		3 & 40\\
		4 & 80\\
		$\vdots$ & $\vdots$
	\end{tabular}
\end{center}

Since you lost count of the ants, you decide to use mathematics to try and figure out
how many ants walked by on minutes $5$, $6$, \ldots. You notice the pattern that
\[
	\text{Green ants per minute $n$} = 2^{n-1}\cdot 10.
\]
Stupendous! Mathematics now predicts there were $160$ ants during minute $5$. But something
else catches your eye. Across the sidewalk are \emph{brown} ants. You count these
ants every minute.

\begin{center}
	\begin{tabular}{c|c}
		Minute & \#Brown Ants\\
		\hline
		1 & 3\\
		2 & 6\\
		3 & 12\\
		4 & 24\\
		$\vdots$ & $\vdots$
	\end{tabular}
\end{center}

The pattern is slightly different. This time, 
\[
	\text{Green ants per minute $n$} = 2^{n-1}\cdot 3.
\]

Your friend, who was watching you the whole time, looks confused. ``Why come up with two complicated equations
when you can describe both types of ant at once?'' they declare.

\begin{center}
	\begin{tabular}{c}
		$\text{\#Ants at minute $n$}\ =\ 2\cdot(\text{\#Ants at minute $n-1$})$\\
		$\text{\#Green ants at minute 1}=10$\\
		$\text{\#Brown ants at minute 1}=3$\\
	\end{tabular}
\end{center}

Your friend has a point. Their model is elegant, but \emph{your} model can predict how many ants pass by at minute $3.222$!
Though, your friend would probably complain that $46.654$ is not a number of ants\ldots.

\medskip

You and your friend have just come up with two different \emph{mathematical models} for the number of ants
that walk across the sidewalk. They happen to make similar predictions for each minute and each have their
strengths and weakenesses. In this course, we will be focused on a particular type of mathematical model---one
that uses \emph{differential equations} at its core.

\Heading{Types of Models}

\begin{definition}[Mathematical Model]
A \emph{mathematical model}\index{model} is a description of the world
	\begin{enumerate}
		\item created in the service of answering a question, and
		\item where the complexity of the world has been abstracted away to numbers, quantities, and their relationships\footnote{ Other
mathematical objects are also allowed.}.
	\end{enumerate}
\end{definition}

In the previous situation, the \emph{question} you were trying to answer was ``how many ants are there at a given minute?''.
We sidestepped difficult issues like, ``Is an ant that is missing three legs still an ant?'' by using the common-sense
convention that ``the number of ants is a whole number and one colored blob that moves under its own power corresponds to one ant''; thus,
we could use single numbers to represent our quantity of interest (the ants).

You and your friend already came up with two types of models.
\begin{itemize}
	\item An \textbf{analytic} model based on known functions.
	\item A \textbf{recursive} model where subsequent terms are based on previous terms and initial conditions.
\end{itemize}

If we define $A(n)$ to be the number of ants corssing the sidewalk at minute $n$, the \emph{analytic} model presented for green ants is
\[
	A(n)=2^{n-1}\cdot 10
\]
and the \emph{recursive} model presented is
\begin{align*}
	A(1) &= 10\\
	A(n) &= 2\cdot A(n-1).
\end{align*}

Each type of model has pros and cons. For example, the analytic model allows you to calculate the number of ants at any minute
with few button presses on a calculator, whereas the recursive model is more difficult to calculate but
makes it clear that the number of ants is doubling every minute.

Often times recursive models are easier to write down than analytic models\footnote{ In fact, in many real-world situations, an analytic model
doesn't exist}, but they maybe harder to analyze. A third type of model has similarities to both analytic and recursive models, and
brings the power of calculus to modeling.

\begin{itemize}
	\item A \textbf{differential-equations} model is a model based on a relationship between a function's derivative(s), its values, and initial conditions.
\end{itemize}

\emph{Differential-equations} models are useful because derivatives correspond to rates of change---and things in the world are always changing.
Let's try to come up with a differential equations model for the ants.

We'd like an equation relating $A(n)$, the number of ants at minute $n$, to $A'(n)$, the \emph{instantaneous rate of change} of the number of ants at minute $n$.
Making a table, we see

\begin{center}
	\begin{tabular}{c|c|c}
		Minute & \#Brown Ants & Change (from prev. minute)\\
		\hline
		1 & 3 & ?\\
		2 & 6 & 3\\
		3 & 12 & 6\\
		4 & 24 & 12\\
	\end{tabular}
\end{center}

or

\begin{center}
	\begin{tabular}{c|c|c}
		Minute & \#Brown Ants & Change (from next minute)\\
		\hline
		1 & 3 & 3\\
		2 & 6 & 6\\
		3 & 12 & 12\\
		4 & 24 & ?\\
	\end{tabular}
\end{center}

depending on whether we record the change from the previous minute or up to the subsequent minute. Neither table gives the \emph{instantaneous} rate of
change, but in both tables, the change is proportional to the number of ants. So, we can set up a model
\[
	A'(n) = kA(n)
\]
where $k$ is a constant of proportionality that we will try to determine later. We've just written down a \emph{differential equation} with an undetermined parameter, $k$.

\begin{definition}[Differential Equation]
	A \emph{differential equation}\index{differential equation} is an equation relating a function to one or more of its derivatives.
\end{definition}

We'd like to figure out what $k$ is. One way to do so is to solve the differential equation and find the values of $k$ so that our model
correctly predicts the data. This is called \emph{fitting} the model to data.

\begin{definition}[Fitting a Model]
	Given a model $M$ with parameters $k_1$, $k_2$, $\ldots$ and data $D$, \emph{fitting the model $M$ to the data $D$}
	is the process of finding values for the parameters $k_1$, $k_2$, $\ldots$ so that $M$ most accurately predicts the data $D$.
\end{definition}

Note that, in general, fitting a model to data doesn't necessarily produce \emph{unique} values for the unknown parameters, and a fitted model
(especially when the data comes from real-world observations) usually doesn't reproduce the data exactly. However, in the case of these ants, we
just might get lucky.

\Heading{Solving Differential Equations}

In general, \emph{there is no algorithm for solving differential equations}. Fortunately, it is easy to check whether any particular
function is a solution to a differential equation, since there \emph{is} an algorithm to differentiate functions\footnote{ More specifically, there is an algorithm
to differentiate the \emph{elementary} functions, those functions formed by compositions, sums, products, and quotients of polynomials, trig, exponentials, and logs.}.
Because of this, \emph{guess and check} will be our primary method for solving differential equations.

\begin{example}
Use educated guessing to solve $A'(n)=kA(n)$.

Since $A'\approx A$, we might start with a function that is equal to its own derivative. There is a famous one: $e^n$. Testing, we see
\[
	\frac{\d}{\d n}e^n=e^n=ke^n
\]
if $k=1$, but it doesn't work for other $k$'s. Trying $e^{kn}$ instead yields
\[
	\frac{\d}{\d n}e^{kn}=ke^{kn}
\]
which holds for all $k$. Thus $A(n)=e^{kn}$ is \emph{a} solution to $A'(n)=kA(n)$. However, there are other solutions, because
\[
	\frac{\d}{\d n}Ce^{kn}=C\left(ke^{kn}\right)=k\left(Ce^{kn}\right),
\]
and so for every $C$, the function $A(n)=Ce^{kn}$ is a solution to $A'(n)=kA(n)$.
\end{example}

By guessing-and-checking, we have found an infinite number of solutions to $A'(n)=kA(n)$.  It's now time to fit our
model to the data.

\begin{example}
	Find values of $C$ and $k$ so that $A(n)=Ce^{kn}$ best models brown ants.

	Taking two rows from our brown ants table, we see
	\begin{align*}
		A(1) &= Ce^{k} = 3\\
		A(2) & = Ce^{2k}=6.
	\end{align*}

	Since $e^k$ can never be zero, from the first equation we get $C=3/e^k$. Combining with the second equation we find
	\[
		Ce^{2k}=\frac{3}{e^k}e^{2k}=3e^k=6
	\]
	and so $e^k=2$. In other words $k=\ln 2$. Plugging this back in, we find $C=3/2$. Thus our fitted model is
	\[
		A(n)=\tfrac{3}{2}e^{n\ln 2}.
	\]
\end{example}

Upon inspection, we can see that $\tfrac{3}{2}e^{n\ln 2} = 3\cdot 2^{n-1}$, which is the analytic model
that was first guessed for brown ants.



%One of the beauties of the guess-and-check method is that it \emph{cannot} produce an incorrect answer,
%since you're always checking (though it may fail to produce answers, any answer that it \emph{does} produce
%will be correct). Thus, even if you did some mathematically-dubious steps to produce a guess, if it passes the checks,
%the final answer is valid.
%
%\begin{example}
%Use Leibniz notation, integral calculus, and guess-and-check to come up with solutions to $A'(n)=kA(n)$.
%
%In Leibniz notation, $A'(n)$ is written as $\displaystyle\frac{\d A}{\d n}$, and we may suppress the function variable and write $A$ in place of $A(n)$. Thus, our equation becomes
%\[
%	\frac{\d A}{\d n}=kA.
%\]
%Treating $\d A/\d n$ as an actual ratio, we can bring all $A$'s to the left and $n$'s to the right, giving
%\[
%	\frac{1}{A}\d A = k\d n.
%\]
%Integrating both sides gives
%\[
%	\int \frac{1}{A}\d A = \ln A + C_1 \qquad =\qquad  \int k\d n = kn + C_2,
%\]
%and so
%\[
%	\ln A + C_1 = kn + C_2 \qquad \iff \qquad \ln A = kn + (C_2-C_1) = kn + C_3.
%\]
%Solving for $A$ by exponentiating both sides gives
%\[
%	A = e^{\ln A} = e^{kn + C_3} = C_4 e^{kn}.
%\]
%Much of what we've done in the preceding calculation \emph{was not rigorous mathematics}, but that's okay, because
%it leads us to the guess: $A(n)=Ce^{kn}$. Finally, we can test our guess and verify
%\[
%	\frac{\d}{\d n}Ce^{kn}=C\left(ke^{kn}\right)=k\left(Ce^{kn}\right),
%\]
%and so $A(n)=Ce^{kn}$ is a solution for every $C$.\footnote{ Notice that in our ``guess work'', $C_4$ was actually restricted to be a positive
%number, but our solution works with $C$ being positive or negative.}
%
%\end{example}



